%%%%%%%%%%%%%%%%%%%%%%%%%%%%%%%%%%%%%%%%%%%%%%%%%%%%%%%%%%%%%%%%%%%%%%%%%%%%%%%%%%
%%%%  UTA Honors College Dissertation Document Generation Using Tex/Latex  
%%%% Adapted from https://github.com/isura/utathesis
%%%%%%%%%%%%%%%%%%%
%%%%%%%%%%%%%%%%%%%%%%%%%%%%%%%%%%%%%%%%%%%%%%%%%%%%%%%%%%%%%%%%%%%%%%%%%%%%%%%%%%
%     Tex/Latex is one of the widely used software to generate many technical and other
%     documents including thesis/dissertations at many universities including UTA.
%
%     The advantage of using Tex/Latex is that with proper style files, most of the
%     front matter required by the university is generated with very little
%     effort (including table of contents, copyright page, list of figures, etc.).
%     All other formatting (margins, fonts, footnote style, equation style,
%     table style, etc.) is built into the style file. With proper style files,
%     it is possible to convert many of the material in any Tex/Latex document
%     so that the generated output fulfills the requirements of an outside publication.
%     Many of the outside publications provide such style files for the conversion.
%
%     The utathesis.zip contains all the sample files that are necessary to generate
%     a typical UTA Ph.D. Dissertation Document including utathesis.sty (contains
%     all the formatting required by UTA Graduate School), graphics.sty, psfig.sty
%     (to include figures that may be in the thesis), amsmath.sty (helpful in generating
%     more complex equations), two typical chapters (which contain equations, figures, and
%     tables, references), two appendices, typical dedication, acknowledgement, abstract,
%     and biographic information files. It is suggested that you unzip utathesis.zip
%     into a new directory so that you can run utaexample.tex contained in
%     the directory to generate a sample output. The utaexample.tex is like a script file
%     which you can modify to suit individual requirements and any minor changes that
%     are required. By commenting out certain portions of the utaexample.tex, it is
%     possible to generate truncated output (just one chapter without front matter, etc.).
%
%     Any Tex/Latex Package including MiKTex (http://www.miktex.org)
%     and Convenience Tex/Latex Editors (http://www.miktex.org/links.html) may be used
%     to generate the Dissertation Document.
%
%     This utaexample.tex file was created by UTA EE Department and uses the
%     bibliography/reference style used by IEEE and adopted by EE Department.
%     The bibliography/reference file acceptable to other departments at UTA are available
%     and must be substituted in the appropriate place.
%
%     Any comments/suggestions  you may have may be sent to prabhu@uta.edu.
%
%%%%%%%%%%%%%%%%%%%%%%%%%%%%%%%%%%%%%%%%%%%%%%%%%%%%%%%%%%%%%%%%%%%%%%%%%%%%%%%%%%

%%%%%%%%%%%%%%%%%%%%%%%%%%%%%%%%%%%%%%%%%%%%%%%%%%%%%%%%%%%%%%%%%%%%%%%%%%%%%%%%%%
%%%%  UTA Ph.D. Dissertation Document Generation Using Tex/Latex   %%%%%%%%%%%%%%%%%%%
%%%%%%%%%%%%%%%%%%%%%%%%%%%%%%%%%%%%%%%%%%%%%%%%%%%%%%%%%%%%%%%%%%%%%%%%%%%%%%%%%%

%%%%%%%%%%%%%%%%%%%%%%%%%%%%%%%%%%%%%%%%%%%%%%%%%%%%%%%%%%%%%%%%%%%%%%%%%%%%%%%%%%
%%%%%%%%%%%%%%%%         all the preamble material            %%%%%%%%%%%%%%%%%%%%
%%%%%%%%%%%%%%%%%%%%%%%%%%%%%%%%%%%%%%%%%%%%%%%%%%%%%%%%%%%%%%%%%%%%%%%%%%%%%%%%%%

\documentclass[12pt]{report}
      \usepackage{utathesis}
%%%%%%%%%%%%%%%%%%%%%%%%%%%%%%%%%%%%%%%%%%%%%%%%%%%%%%%%%%%%%%%%%%%%%%%%%%%%%%%%%%
%%%%%%%%%%%%%%%%%%%%   load any packages which are needed   %%%%%%%%%%%%%%%%%%%%%%
%%%%%%%%%%%%%%%%%%%%%%%%%%%%%%%%%%%%%%%%%%%%%%%%%%%%%%%%%%%%%%%%%%%%%%%%%%%%%%%%%%
%     \usepackage{latexsym} % to get LASY symbols
%     \usepackage{graphicx} % to insert PostScript figures
%     \usepackage{rotating} % for sideways tables/figures
%%%%%%%%%%%%%%%%%%%%%%%%%%%%%%%%%%%%%%%%%%%%%%%%%%%%%%%%%%%%%%%%%%%%%%%%%%%%%%%%%%
%%%%%%%%%%%%%%%%         all the preamble material            %%%%%%%%%%%%%%%%%%%%
%%%%%%%%%%%%%%%%%%%%%%%%%%%%%%%%%%%%%%%%%%%%%%%%%%%%%%%%%%%%%%%%%%%%%%%%%%%%%%%%%%
      \usepackage{graphicx}
      \usepackage{amsmath}
    \begin{document}
    \input{psfig.sty}
%     SAMPLE FRONT MATTER:
       \graduationmonth{May}
       \graduationyear{2019}
       \defensedate{April 12, 2019}
       \author{Author Name}
       \committee{Professor Name}
       \title{TITLE}
%%%%%%%%%%%%%%%%%%%%%%%%%%%%%%%%%%%%%%%%%%%%%%%%%%%%%%%%%%%%%%%%
%%%%%%%%%%%%%%  copyright page  %%%%%%%%%%%%%%%%%%%%%%%%%%%%%%%%
%%%%%%%%%%%%%%%%%%%%%%%%%%%%%%%%%%%%%%%%%%%%%%%%%%%%%%%%%%%%%%%%

         \copyrightpage

%%%%%%%%%%%%%%%%%%%%%%%%%%%%%%%%%%%%%%%%%%%%%%%%%%%%%%%%%%%%%%%%
%%%%%%%%%%%%%%  title page  %%%%%%%%%%%%%%%%%%%%%%%%%%%%%%%%
%%%%%%%%%%%%%%%%%%%%%%%%%%%%%%%%%%%%%%%%%%%%%%%%%%%%%%%%%%%%%%%%

          \titlepage

% If submitting electronically, the signature page should not be included.
% You should then comment out the line that produces the signature page and
% move the title page up here so that it appears first.
% added by Darin Brezeale, Fri Jan 11 10:29:12 CST 2008
%       \signaturepageMs
%        \signaturepagePhD


%%%%%%%%%%%%%%%%%%%%%%%%%%%%%%%%%%%%%%%%%%%%%%%%%%%%%%%%%%%%%%%
%%%%%%%%%%%%%%  acknowledgements  %%%%%%%%%%%%%%%%%%%%%%%%%%%%%%%%
%%%%%%%%%%%%%%%%%%%%%%%%%%%%%%%%%%%%%%%%%%%%%%%%%%%%%%%%%%%%%%%%

\begin{acknowledgements}

     \input acknowledge.tex

\end{acknowledgements}
%%%%%%%%%%%%%%%%%%%%%%%%%%%%%%%%%%%%%%%%%%%%%%%%%%%%%%%%%%%%%%%
%%%%%%%%%%%%%%     abstract    %%%%%%%%%%%%%%%%%%%%%%%%%%%%%%%%
%%%%%%%%%%%%%%%%%%%%%%%%%%%%%%%%%%%%%%%%%%%%%%%%%%%%%%%%%%%%%%%%
\begin{abstract}
	\input abstract.tex

	\indent
\end{abstract}
\tableofcontents
\listoffigures
\listoftables
\addtocontents{toc}{\noindent\mbox{Chapter}\hfill\mbox{Page}}%
%\addtocontents{toc}{\noindent\mbox{Chapter}}%
%%%%%%%%%%%%%%%%%%%%%%%%%%%%%%%%%%%%%%%%%%%%%%%%%%%%%%%%%%%%%%%
%%%%%%%%%%%%%%  First and Following Chapters  %%%%%%%%%%%%%%%%%%
%%%%%%%%%%%%%%%%%%%%%%%%%%%%%%%%%%%%%%%%%%%%%%%%%%%%%%%%%%%%%%%%
%     \input macros.tex      % file containing author's macro definitions
\input chap1.tex      % file containing Chapter 2 contents
\input chap2.tex      % file containing Chapter 2 contents
%%%%%%%%%%%%%%%%%%%%%%%%%%%%%%%%%%%%%%%%%%%%%%%%%%%%%%%%%%%%%%%%%%%%%%%%%%%%%%%%%%
%%%%%%%%%%%%%%%%                  Appendices                  %%%%%%%%%%%%%%%%%%%%
%%%%%%%%%%%%%%%%%%%%%%%%%%%%%%%%%%%%%%%%%%%%%%%%%%%%%%%%%%%%%%%%%%%%%%%%%%%%%%%%%%
\appendix
%      \chapter{FIRST APPENDIX NAME}
\chapter{JENSEN'S INEQUALITY FOR CONVEX FUNCTIONS}
\input appA.tex       % file with Appendix A contents
\chapter{UPPER BOUNDS ON MOMENTS}
\input appB.tex       % file with Appendix B contents
%%%%%%%%%%%%%%%%%%%%%%%%%%%%%%%%%%%%%%%%%%%%%%%%%%%%%%%%%%%%%%%%%%%%%%%%%%%%%%%%%%
%%%%%%%%%%%%%%%%                 Bibliography                 %%%%%%%%%%%%%%%%%%%%
%%%%%%%%%%%%%%%%%%%%%%%%%%%%%%%%%%%%%%%%%%%%%%%%%%%%%%%%%%%%%%%%%%%%%%%%%%%%%%%%%%

%\bibliographystyle{plain}
%\bibliography{test}
%\bibliography{alpha}
\bibliographystyle{IEEEtran} % use IEEEtran.bst style
%\nocite{*}                   % list all refs in database, cited or not
%bibliography{refs}           % bib database file refs.bib
\bibliography{IEEEabrv,IEEEexample1}

%%%%%%%%%%%%%%%%%%%%%%%%%%%%%%%%%%%%%%%%%%%%%%%%%%%%%%%%%%%%%%%%%%%%%%%%%%%%%%%%%%
%%%%%%%%%%%%%%%%         Biographical Statement               %%%%%%%%%%%%%%%%%%%%
%%%%%%%%%%%%%%%%%%%%%%%%%%%%%%%%%%%%%%%%%%%%%%%%%%%%%%%%%%%%%%%%%%%%%%%%%%%%%%%%%%
\thebiography
\input biography.tex

\end{document}

